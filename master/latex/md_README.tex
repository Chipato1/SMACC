\tabulinesep=1mm
\begin{longtabu} spread 0pt [c]{*3{|X[-1]}|}
\hline
\rowcolor{\tableheadbgcolor}{\bf R\+OS Distro }&{\bf Travis Build Status }&{\bf Documentation  }\\\cline{1-3}
\endfirsthead
\hline
\endfoot
\hline
\rowcolor{\tableheadbgcolor}{\bf R\+OS Distro }&{\bf Travis Build Status }&{\bf Documentation  }\\\cline{1-3}
\endhead
Indigo & &\href{https://reelrbtx.github.io/SMACC/indigo-devel/html/md_README.html}{\tt doxygen} \\\cline{1-3}
Kinetic & &\href{https://reelrbtx.github.io/SMACC/kinetic-devel/html/md_README.html}{\tt doxygen} \\\cline{1-3}
Melodic & &\href{https://reelrbtx.github.io/SMACC/melodic-devel/html/md_README.html}{\tt doxygen} \\\cline{1-3}
Master & &\href{https://reelrbtx.github.io/SMACC/master/html/md_README.html}{\tt doxygen} \\\cline{1-3}
\end{longtabu}


\subsection*{Docker Containers}

\href{https://hub.docker.com/r/pabloinigoblasco/smacc/}{\tt } \href{https://hub.docker.com/r/pabloinigoblasco/smacc/}{\tt } \href{https://registry.hub.docker.com/pabloinigoblasco/smacc/}{\tt }

\section*{S\+M\+A\+CC}

S\+M\+A\+CC is an event-\/driven, asynchronous, behavioral state machine library for real-\/time R\+OS (Robotic Operating System) applications written in C++, designed to allow programmers to build robot control applications for multicomponent robots, in an intuitive and systematic manner \href{http://sce.uhcl.edu/helm/rationalunifiedprocess/process/modguide/md_stadm.htm}{\tt U\+ML State Charts} (A\+KA state machines). S\+M\+A\+CC is inspired by the \href{http://wiki.ros.org/smach}{\tt S\+M\+A\+CH R\+OS package} and it is built on top of \href{https://www.boost.org/doc/libs/1_53_0/libs/statechart/doc/index.html}{\tt Boost State\+Chart library}.

Probably the greatest strength of S\+M\+A\+CC is that the state machines you can develop with it are strictly based on the U\+ML Standard. This means that you have access to a clear and thoroughly studied and known approach to describe State Machines. This may be especially important in industrial environments.

The following image shows one example of a state machine built using the U\+ML standard and shows many of the concepts that can be implemented using S\+M\+A\+CC\+: 

 

\subsection*{Features}


\begin{DoxyItemize}
\item $\ast$$\ast$$\ast$\+Powered by R\+OS\+:$\ast$$\ast$$\ast$ S\+M\+A\+CC has been developed specifically to work with R\+OS. It is a c++ ros package that can be imported from any end-\/user application package.
\item $\ast$$\ast$$\ast$\+C++ language\+:$\ast$$\ast$$\ast$ Until now, R\+OS has lacked a library to develop task-\/level state machines in C++. Many libraries in robotics are developed in C++ so this may help during the integration of different libraries. In industrial development contexts C++ is often over Python, so this tool may be a good choice.
\item $\ast$$\ast$$\ast$\+Static State Machine Checking\+:$\ast$$\ast$$\ast$ S\+M\+A\+CC inherits this from the Boost Statechart library. This helps the developer to check the consistence of the state machine at compile time (instead of runtime). In other words, it helps you to check if your state machine is well written.
\item $\ast$$\ast$$\ast$\+Component based architecture\+:$\ast$$\ast$$\ast$ S\+M\+A\+CC has built-\/in funcionality provided inside S\+M\+A\+CC Components that can be dynamically imported at runtime and stored in the local machine. The states only access those components they are concerned with. This enables the S\+M\+A\+CC based application to extend or improve the runtime behavior of the system.
\end{DoxyItemize}

\subsection*{Canonical S\+M\+A\+CC applications}

The canonical S\+M\+A\+CC applications are mobile robots (that may optionally have manipulators) that have to navigate around the environment and use one or more onboard tools. One example of this would be the P\+R2 Robot, working in a factory, navigating to some shelves with parcels, fetching them with a 6-\/axis arm and manipulator, and then navigating to some delivery point. Other examples would include mobile robots that must perform a systematic navigation on a designated area while operating one or more tools, such as a vacuum or angle grinder. In order to accomplish this, S\+M\+A\+CC provides an array navigation planners that can greatly simplifiy this task (see more on section R\+OS Integration).

\subsection*{R\+OS Integration}


\begin{DoxyItemize}
\item $\ast$$\ast$$\ast$\+Intensive use of R\+OS Action$\ast$$\ast$$\ast$. S\+M\+A\+CC translates Action server events (Result callbacks, Feedback callbacks, etc.) into statechart events. To learn more about this, check the sections Shared Resources and S\+M\+A\+CC Architecture.
\item $\ast$$\ast$$\ast$\+Powerful access to R\+OS Parameters$\ast$$\ast$$\ast$. Each S\+M\+A\+CC state automatically creates a ros\+::\+Node\+Handle automatically named according to the S\+M\+A\+CC state hierarchy (see more in section Usage Examples -\/ Ros parameters)
\item $\ast$$\ast$$\ast$\+R\+OS Navigation built-\/in funcionality$\ast$$\ast$$\ast$. S\+M\+A\+CC extends the R\+OS navigation stack in a high level way. It provides specialized navigation planners (for the R\+OS Navigation Stack) that navigate only using pure spinning motions and straight motions. Implements some mechanism to perform motions recording the path and undoing them later. These can be very useful in some industrial applications where the knowledge or certainty on the environment is higher (ros planners are focused on cluttered and dynamic environments).
\end{DoxyItemize}

\subsection*{Repository Packages}

This repository contains several R\+OS packages\+:


\begin{DoxyItemize}
\item {\bfseries smacc\+\_\+core}\+: The core smacc library. It works as a template-\/based c++ header library.
\item {\bfseries smacc\+\_\+navigation}\+: a set of ros packages with some \char`\"{}\+Smacc Components\char`\"{} that ease the creation of high level navigation applications. The provided components remotelly control the move\+\_\+base. These components helps in the creation of complex motion strategies (changing planners, recording and undoing paths, etc.)
\item {\bfseries smacc\+\_\+state\+\_\+machine\+\_\+templates}\+: shows a complete sample application developed with S\+M\+A\+CC that can be reused as a canonical example of a mobile robot moving around and interacting with a custom onboard tool.
\item {\bfseries smacc\+\_\+interface\+\_\+components}\+: template project that shows how an on/off tool onboard the mobile robot could be used following the S\+M\+A\+CC methodology.
\end{DoxyItemize}

\subsection*{Future Work}


\begin{DoxyItemize}
\item undoing paths chunks by state (store the different chunks of the path according to its state in a stack)
\item code generation based on uml diagrams
\item improving backwards planners for non linear paths
\end{DoxyItemize}

\subsection*{Shared Resources and Shared variables}


\begin{DoxyItemize}
\item $\ast$$\ast$$\ast$\+Shared Action Client Resources$\ast$$\ast$$\ast$ Action servers in S\+M\+A\+CC play an important role because all low-\/level funcionality must be located on them. In order to interact with these Action Servers S\+M\+A\+CC provide an easy way to create shared Action\+Clients that can be accessed from any State. S\+M\+A\+CC is in charge of eficently handle all the requests and send the resulting events from action servers (result messages, feedback messages, etc) to the \char`\"{}subscribed states\char`\"{} in form of statechart events. (See more about his in section internal architecture)
\item $\ast$$\ast$$\ast$\+Shared Variables$\ast$$\ast$$\ast$ U\+ML statecharts basically define the high level behavior of a system. However, in practice the real state of the system may be much more complex (mesurement, environment numerical information, etc.). States usually have to share information (or comunicate to each other). In order to do that, S\+M\+A\+CC implements a simple but effective dictionary-\/based mechanism to share information (structs, objects, simple variables or pointers). (See below in tutorials\+: shared variable) 
\end{DoxyItemize}

 

\subsection*{Development methodology}

S\+M\+A\+CC also defines a development methodology where State Machine nodes only contain the task-\/level logic, that is, the high level behavior of the robot system in some specific application.

S\+M\+A\+CC applications have low level coupling with other software components of the robot system. S\+M\+A\+CC code is recomended to interact with the rest of components the robot system via R\+OS Action Servers and {\bfseries Smacc Action Plugins}.

The proposed methdology split the states into 2 or more statechart orthogonal lines that comunicate to each other via events. The orthogonal line 0 is tipically for the mobile robot navigation. The second orthogonal line and ahead are used for tools (manipulators, grippers or other custom tools).

 

\subsection*{Internal Architecture}

S\+M\+A\+CC State Machines are boost\+::statechart Asynchronous\+State\+Machines that can work in a multi-\/threaded application. In S\+M\+A\+CC State Machines are two main components that work concurrently in two different threads\+:


\begin{DoxyItemize}
\item $\ast$$\ast$$\ast$\+Signal Detector$\ast$$\ast$$\ast$. It is able to handle the action client components communication with action servers and translate them to statechart events
\item $\ast$$\ast$$\ast$\+State Machine$\ast$$\ast$$\ast$. It is the end-\/user code of the state machine itself.
\end{DoxyItemize}

 

\subsection*{Executing the Radial Motion Example}

This is a complete sample of a state machine that controls the motion of a simulated ridgeback mobile robot in gazebo.

This example shows how smacc\+::\+Navigation components can be used to create some systematic motion based on an state machine. The robot performs a motion pattern that plots a \char`\"{}\+Start\char`\"{} on the plane. The state machine in the motion is combined with the usage of some external tool \char`\"{}for example ground painter\char`\"{} that is only active on forward motions.


\begin{DoxyCode}
1 export RIDGEBACK\_URDF\_EXTRAS=$(rospack find radial\_motion\_example)/urdf/empty.xacro
2 
3 roslaunch radial\_motion\_example radial\_motion.launch
\end{DoxyCode}
 

      

\section*{Tutorial}

S\+M\+A\+CC states inherits from boost\+::statechart\+:State so that you can learn the full potential of S\+M\+A\+CC states also diving in the statechart documentation. However, the following examples briefly show how you create define S\+M\+A\+CC states and how you would usually use them.

\subsection*{Code a minimal S\+M\+A\+CC State\+Machine}

In this initial example we will implement a simple state machine with a single state state that executes something at state entry and at state exit. That state machine is described in the following image\+:

 

S\+M\+A\+CC State\+Machines and Smacc\+States are based on the c++ \href{https://en.wikipedia.org/wiki/Curiously_recurring_template_pattern}{\tt Curiously recurring template pattern} so that the syntax may be strange for some developers but you will notice that it is very easy to follow. The advantage of using this kind of c++ pattern is that the definition of the state machine is correctly written.

The following chunk of code shows the minimal S\+M\+A\+CC State machine you can create\+: 
\begin{DoxyCode}
\textcolor{preprocessor}{#include <\hyperlink{smacc__state__machine__base_8h}{smacc/smacc\_state\_machine\_base.h}>}

\textcolor{keyword}{struct }SimpleStateMachine
    : \textcolor{keyword}{public} SmaccStateMachineBase<SimpleStateMachine,ToolSimpleState>
\{
  \textcolor{keyword}{using} SmaccState::SmaccState;
  \textcolor{keywordtype}{void} onEntry()
  \{
  \}
\};
\end{DoxyCode}


Every S\+M\+A\+CC State Machine must inherit from Smacc\+State\+Machine\+Base. The first template parameter is the derived class type (Simple\+State\+Machine itself) according to the curiosuly recurring template pattern, and the second template parameter (Tool\+Simple\+State) would be the class name of the initial Smacc State

\subsection*{Code a simple S\+M\+A\+CC State}

For the previous state machine, this would be the initial S\+M\+A\+CC State. It also follows the Curiously recurrent template pattern. However, for Smacc states, the second template parameters is the so called \char`\"{}\+Context\char`\"{}, for this simple case, the context is the State\+Machine type itself. However, that could also be other State (in a nexted-\/substate case) or an orthogonal line.


\begin{DoxyCode}
\textcolor{keyword}{struct }ToolSimpleState
    : \hyperlink{classSmaccState}{SmaccState}<ToolSimpleState, SimpleStateMachine>
\{
\textcolor{keyword}{public}:

  \textcolor{keyword}{using} SmaccState::SmaccState;
  \textcolor{keywordtype}{void} onEntry()
  \{
    ROS\_INFO(\textcolor{stringliteral}{"Entering ToolSimpleState"});
  \}
\};

\textcolor{keywordtype}{int} \hyperlink{namespacerosdoc__lite_af17b2a5046f7827add27b1cc8deb06a0}{main}(\textcolor{keywordtype}{int} argc, \textcolor{keywordtype}{char} **argv) \{
  \textcolor{comment}{// initialize the ros node}
  ros::init(argc, argv, \textcolor{stringliteral}{"example1"});
  ros::NodeHandle nh;

  smacc::run<SimpleStateMachine>();
\}
\end{DoxyCode}
 According to the U\+ML statchart standard, things happens essencially when the system enters in the state, when the system exits the state and when some event is triggered. The two first ones are shown in this example. The c++ Constructor code is the place you have to write your \char`\"{}entry code\char`\"{}, the destructor is the place you have to write your \char`\"{}exit code\char`\"{}. The constructor parameter (my\+\_\+context) is a reference to the context object (in this case the state machine). This kind of constructor may be verebosy, but is required to implement the rest of S\+M\+A\+CC tasks and always follows the same pattern.

\subsection*{Creating/accessing to Smacc\+Components}

Accessing to S\+M\+A\+CC componets resources is one of the most important capabilities that S\+M\+A\+CC provides. This example shows how to access to these resources form states.

For example, in this case we will asume we are in a state that controls the navigation of the vehicle, and it needs to access to the Ros Navigation Stack Action client and navigate to some position in the environment.

 

The code would be the following\+:


\begin{DoxyCode}
\textcolor{keyword}{struct }Navigate : \hyperlink{classSmaccState}{SmaccState}<Navigate, SimpleStateMachine> 
\{

\textcolor{keyword}{public}:
  \textcolor{comment}{// This is the smacc component (it basically is a wrapper of the ROS Action Client for move base), please
       check the SMACC}
  \textcolor{comment}{// code to see how to implement your own action client}
  \hyperlink{classsmacc_1_1SmaccMoveBaseActionClient}{smacc::SmaccMoveBaseActionClient} *moveBaseClient\_;

  \textcolor{keyword}{using} SmaccState::SmaccState;
  \textcolor{keywordtype}{void} onEntry()
  \{
    ROS\_INFO(\textcolor{stringliteral}{"Entering Navigate"});

    \textcolor{comment}{// this substate will need access to the "MoveBase" resource or plugin. In this line}
    \textcolor{comment}{// you get the reference to this resource.}
    this->requiresComponent(moveBaseClient\_);
    goToEndPoint();
  \}

  \textcolor{comment}{// auxiliar function that defines the motion that is requested to the move\_base action server}
  \textcolor{keywordtype}{void} goToEndPoint() \{
    geometry\_msgs::PoseStamped radialStartPose = createInitialPose();

    smacc::SmaccMoveBaseActionClient::Goal goal;
    goal.target\_pose.header.stamp = ros::Time::now();

    goal.target\_pose = radialStartPose;
    goal.target\_pose.pose.position.x = 10;
    goal.target\_pose.pose.position.y = 10;
    goal.target\_pose.pose.orientation =
        tf::createQuaternionMsgFromRollPitchYaw(0, 0, M\_PI);

    moveBaseClient\_->\hyperlink{classsmacc_1_1SmaccActionClientBase_a2ec9e5fb96ecc517a815cda209eb1b51}{sendGoal}(goal);
  \}
\};
\end{DoxyCode}


\subsection*{Simple State Transition on Action Result Event}

According to the U\+ML state machines standard, transitions between states happen on events. In S\+M\+A\+CC events can be implemented by the user or happen when Action Results callbacks and Action Feedback callbacks happen. In the following example we extend the previous example to transit to another state \textquotesingle{}Execute\+Tool\+State\textquotesingle{} when the move\+\_\+base action sever returns a Result.

 

The following would be the code to implement the diagram shown above.


\begin{DoxyCode}
\textcolor{keyword}{struct }Navigate : \hyperlink{classSmaccState}{SmaccState}<Navigate, SimpleStateMachine>
\{
\textcolor{keyword}{public}:

  \textcolor{comment}{// With this line we specify that we are going to react to any EvActionResult event}
  \textcolor{comment}{// generated by SMACC when the action server provides a response to our request}
  \textcolor{keyword}{typedef} mpl::list<sc::transition<EvActionResult<SmaccMoveBaseActionClient::Result>, ExecuteToolState>> 
      reactions;

  \textcolor{keyword}{using} SmaccState::SmaccState;
  \textcolor{keywordtype}{void} onEntry()
  \{
   [...]
  \}
\};

\textcolor{keyword}{struct }ExecuteToolState : \hyperlink{classSmaccState}{SmaccState}<ExecuteToolState, SimpleStateMachine>
\{
    \textcolor{keyword}{using} SmaccState::SmaccState;
    \textcolor{keywordtype}{void} onEntry()
    \{
    \}
\};
\end{DoxyCode}


\subsection*{Add custom code on Action Result Events}

In the following example, we want to add some code to the transition between the source state \char`\"{}\+Navigate\char`\"{} and the destination state \char`\"{}\+Execute\+Tool\+State\char`\"{}. This code may be any desired custom code (for example some transition guard). This code is located in the react method

 

The following would be the code for this state machine\+:


\begin{DoxyCode}
\textcolor{keyword}{struct }Navigate : \hyperlink{classSmaccState}{SmaccState}<Navigate, SimpleStateMachine> 
\{
\textcolor{keyword}{public}:

  \textcolor{comment}{// With this line we specify that we are going to react to any EvActionResult event}
  \textcolor{comment}{// generated by SMACC when the action server provides a response to our request}
  \textcolor{keyword}{typedef} mpl::list<sc::custom\_reaction<EvActionResult<SmaccMoveBaseActionClient::Result>>> reactions;
  \textcolor{keyword}{using} SmaccState::SmaccState;
  \textcolor{keywordtype}{void} onEntry()
  \{
   [...]
  \}

  \textcolor{comment}{// auxiliary function that defines the motion that is requested to the move\_base action server}
  \textcolor{keywordtype}{void} goToEndPoint() \{
   [...]

  sc::result react(\textcolor{keyword}{const} EvActionResult<SmaccMoveBaseActionClient::Result> &ev)
  \{
      \textcolor{comment}{// we only will react when the result is succeeded}
      \textcolor{keywordflow}{if} (ev.getResult() == actionlib::SimpleClientGoalState::SUCCEEDED)
      \{
        \textcolor{comment}{// ev.resultMessage provides access to the move\_base action server result structure}

        ROS\_INFO(\textcolor{stringliteral}{"Received event to movebase: %s"},ev.getResult().toString().c\_str());
        \textcolor{keywordflow}{return} transit<ExecuteToolState>();
      \}
      \textcolor{keywordflow}{else}
      \{
        \textcolor{keywordflow}{return} forward\_event(); \textcolor{comment}{// do nothing, the default behavior if you do not specify any return value}
      \} 
  \}
\};

\textcolor{keyword}{struct }ExecuteToolState : \hyperlink{classSmaccState}{SmaccState}<ExecuteToolState, SimpleStateMachine> 
\{
    \textcolor{keyword}{using} SmaccState::SmaccState;
    \textcolor{keywordtype}{void} onEntry()
    \{
    \}
\};
\end{DoxyCode}


\subsection*{Adding R\+OS Parameters to Smacc States}

The S\+M\+A\+CC states can be configured from the ros parameter server based on their hierarchy and their class name. It is responsability of the user not to have two different state names at the same level (even if the namespace is distinct since the namespace is trimmed for parameters)

For example, imagine a State\+Machine to move the mobile robot initially to some initial position, and then moving it to some other position relative to the initial position. You could put the navigation parameters in a ros configuration yaml file like this (and avoid hardcoding)\+:


\begin{DoxyCode}
1 MyStateMachine:
2     State1: #Go to some initial position
3         NavigationOrthogonalLine:
4             Navigate:
5                 start\_position\_x: 3
6                 start\_position\_y: 0
7     State2: #Go to some initial position
8         NavigationOrthogonalLine:
9             Navigate:
10                 initial\_orientation\_index: 0 # the initial index of the linear motion (factor of
       angle\_increment\_degrees)
11                 angle\_increment\_degree: 90    # the increment of angle between to linear motions
12                 linear\_trajectories\_count: 2  # the number of linear trajectories of the radial 
\end{DoxyCode}


Then, the c++ code for the State My\+State\+Machine/\+State1/\+Navigation\+Orthogonal\+Line/\+Navigate could contain the following parameter reading funcionality\+:


\begin{DoxyCode}
\textcolor{keyword}{struct }Navigate : \hyperlink{classSmaccState}{SmaccState}<Navigate, NavigationOrthogonalLine> 
\{
\textcolor{keyword}{public}:
  \textcolor{keyword}{using} SmaccState::SmaccState;
  \textcolor{keywordtype}{void} onEntry()
  \{
      geometry\_msgs::Point p;
      param(\textcolor{stringliteral}{"start\_position\_x"}, p.x, 0);
      param(\textcolor{stringliteral}{"start\_position\_y"}, p.y, 0);
  \}
\}
\end{DoxyCode}


The param template method reads from the parameters server delegating to the method defined ros\+::\+Node\+Handle handle does but already located at the exact point in the parameter name hierarchy associated to this state. S\+M\+A\+CC is also able have methods get\+Param and set\+Param that are delegated to ros\+::\+Node\+Handle in the same way.

\subsection*{Shared variables between states}

This following example shows how to share a variable between two states. In the Navigate state the \char`\"{}angle\+\_\+value\char`\"{} variable is set using the template method \char`\"{}set\+Data\char`\"{}. Latter in the Execute\+Tool\+State it gets the value using the template method \char`\"{}get\+Global\+S\+M\+Data\char`\"{}.


\begin{DoxyCode}
\textcolor{keyword}{struct }Navigate : \hyperlink{classSmaccState}{SmaccState}<Navigate, SimpleStateMachine>
\{
\textcolor{keyword}{public}:
  \textcolor{keyword}{using} SmaccState::SmaccState;
  \textcolor{keywordtype}{void} onEntry()
  \{
        \textcolor{keywordtype}{double} angle = M\_PI;
        this->setGlobalSMData(\textcolor{stringliteral}{"angle\_value"}, angle);
  \}
\};

\textcolor{keyword}{struct }ExecuteToolState : \hyperlink{classSmaccState}{SmaccState}<ExecuteToolState, SimpleStateMachine>
\{
    \textcolor{keyword}{using} SmaccState::SmaccState;
    \textcolor{keywordtype}{void} onEntry()
    \{
         \textcolor{keywordtype}{int} angle;
         this->getGlobalSMData(\textcolor{stringliteral}{"angle\_value"}, angle);
    \}
\};
\end{DoxyCode}


\subsection*{Orthogonal Lines}

S\+M\+A\+CC proposes to work in different orthogonal lines\+: Navigation, Tool1, Tool2, etc. This example shows how you can define orthogonal lines in your S\+M\+A\+CC code. For example, we want to add two orthogonal lines\+: the navigation orthogonal line and the tool orthogonal line.

 

First we will define the \hyperlink{classNavigationOrthogonal}{Navigation\+Orthogonal} line line with a simple Tool\+Sub\+State\+:


\begin{DoxyCode}
\textcolor{keyword}{struct }NavigationOrthogonalLine
    : \hyperlink{classSmaccState}{SmaccState}<NavigationOrthogonalLine, SimpleStateMachine::orthogonal<0>, NavigateSubstate> \{
  \textcolor{keyword}{using} SmaccState::SmaccState;
  \textcolor{keywordtype}{void} onEntry()                 
  \{
    ROS\_INFO(\textcolor{stringliteral}{"Entering in the navigation orthogonal line"});
  \}

  ~NavigationOrthogonalLine() 
  \{
    ROS\_INFO(\textcolor{stringliteral}{"Finishing the navigation orthogonal line"});
  \}
\};

\textcolor{keyword}{struct }NavigateSubstate : \hyperlink{classSmaccState}{SmaccState}<NavigateSubstate, NavigationOrthogonalLine> 
\{
    \textcolor{keyword}{using} SmaccState::SmaccState;
    \textcolor{keywordtype}{void} onEntry()
    \{
    \}
\};
\end{DoxyCode}
 First we will define the Tool\+Othogonal line with a simple Tool\+Sub\+State\+:


\begin{DoxyCode}
\textcolor{keyword}{struct }ToolOrthogonalLine
    : \hyperlink{classSmaccState}{SmaccState}<ToolOrthogonalLine, SimpleStateMachine::orthogonal<1>, ToolSubstate> \{
\textcolor{keyword}{public}:
  \textcolor{keyword}{using} SmaccState::SmaccState;
  \textcolor{keywordtype}{void} onEntry()          
  \{
    ROS\_INFO(\textcolor{stringliteral}{"Entering in the tool orthogonal line"});
  \}

  ~ToolOrthogonalLine() 
  \{
    ROS\_INFO(\textcolor{stringliteral}{"Finishing the tool orthogonal line"}); 
  \}
\};

\textcolor{keyword}{struct }ToolSubstate : \hyperlink{classSmaccState}{SmaccState}<ToolSubstate, ToolOrthogonalLine> 
\{
    \textcolor{keyword}{using} SmaccState::SmaccState;
    \textcolor{keywordtype}{void} onEntry()
    \{
    \}
\};
\end{DoxyCode}
 